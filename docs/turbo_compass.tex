% !TEX root = /home/eecs/charleschien101/Compass/docs/turbo_compass.tex

\documentclass[12pt]{article}

\usepackage{amsmath}
\usepackage{graphicx}
\usepackage{hyperref}
\usepackage[utf8]{inputenc}
\usepackage{amsfonts}

\title{How to Set Up LaTeX on VSCode}

\author{Mozilla Club of UCSC}

\date{2020–12–10}

\begin{document}

\maketitle

\section{Introduction}

Here is the introduction of our text

Turbo-Compass

Turbo-Compass is based on a \textbf{black-box iterative matrix completion algorithm}. In each step of the algorithm, only a subset of entries of the reaction score matrix are computed and used for imputation of the entire reaction score matrix, allowing for faster runtime at the expense of accuracy.

Initially, $p$ percent of all matrix entries are chosen uniformly at random for computation; $p$ is a hyperparameter of the method with default value $p = 0.01$ (corresponding to 1\% sampling of the matrix). Next, given the entries observed so far, a model is used to impute the remaining entries. The quality of the imputations is measured by 5-fold cross-validation: 80% of the observed entries are used to fit the model, and the remaining 20% are used to evaluate model fit. This way, we obtain a Spearman correlation for each reaction averaged over the 5 folds. The goal of the algorithm is for every reaction to be imputed with a cross-validated Spearman correlation of $\rho$, which is a hyperparameter of the method, by default $\rho = 0.95$ (corresponding to 95%). If all reactions meet this threshold, the algorithm terminates and returns the imputed reaction score matrix (by first re-fitting the model on all the available data, rather than using a specific cross-validation split, which would be wasteful). Otherwise, we query another $p$ percent of the entries of the matrix, distributing the budget evenly among the reactions failing to meet the threshold. Any remaining budget is evenly distributed among the well-imputed reactions. Cross-validated Spearman correlation is then re-evaluated, and the process is iterated until all reactions meet the cutoff $\rho$. Thus, if the algorithm performs t iterations in total, only a fraction $tp$ of the entries in the reaction score matrix will have been queried. Typically, $tp$ is much less than 100% (for example 5%), leading to a magnitude order speedup (such as 20×) over naively querying all entries using the vanilla implementation of Compass.

In reality, there might be a handful of reactions that are very hard to predict because they are noisy. To cope with this, we relax the constraint that all reactions must be imputed with a cross-validated Spearman correlation of $\rho$. Instead, we require that *at least* $q$ percent of the reactions meet the threshold; $q$ is a hyperparameter of the method, by default $q = 0.99$ (corresponding to 99%).

The last but most important component of the method is the model used to impute entries of the reaction score matrix $X \in \mathbb{R}^{m \times n}$. We use a low-rank matrix completion model. This means we estimate low-rank matrices $A \in \mathbb{R}^{m \times k}, B \in \mathbb{R}^{n \times k}$, such that $X \approx AB^T$. More precisely, we solve:

$\mathop{\arg \min}\limits_{A \in \mathbb{R}^{m \times k}, B \in \mathbb{R}^{n \times k}} \frac{1}{2} ||P_\Omega(X - AB^T)||^2_F + \frac{\lambda}{2} (||A||^2_F + ||B||^2_F)$

\begin{itemize}

\item One

\item Two

\item Three

\end{itemize}

\end{document}